%%%--------------------------------%%%
%%% Theory
%%%--------------------------------%%%
\newpage
\section{Gamification}
\label{sec:theoryB}

The following chapters aim is to clarify the main theory behind human motivation, gamification and the corresponding patterns and methods. Therefore first of all the term Gamification is defined and explained (chapter \ref{sec:theoryBa}), furthermore there is an introduction to human motivation (chapter \ref{sec:theoryBb}) and motivational patterns (chapter \ref{sec:theoryBc}). TODO REST

\subsection{Definition Gamification}
\label{sec:theoryBa}

The term gamification is defined by Kumar and Herger as follows:

\begin{fquote}[Gamification {\protect\cite[p. 8]{inproceedings}}]
	Gamification is the application of game design principles and mechanics to
	non-game environments. It attempts to make technology more inviting by encouraging users to engage in desired behaviors and by showing the path to mastery.
	From a business viewpoint, gamification is using people’s innate enjoyment of play.
\end{fquote}

Further definitions are TODO

\cite[p. 8]{inproceedings}

More umfassender: Motivation -> next chapter(s)

Gamification aims to motivate the user to do something. That's why the next chapter provides a more comprehensive introduction on motivation.

\subsection{Motivation}
\label{sec:theoryBb}

The game design principles and mechanics which are used in the context of gamification are a specialization of motivational patterns used in Human Computer Interaction. \cite[p. 59]{inproceedings}

Therefore this chapter provides an introduction into the underlying psychology of motivation, different types of motivation (extrinsic and intrinsic) and TODO!?

Human motivation is one of the main areas of psychology. Some questions which arouse are: What motivates humans for doing something? TODO: weiter Fragen (aus Quelle) \cite[p. 1]{bierhoffeditorEnzyklopaediePsychologieSoziale2016}

TODO: weiter nach bierhoffeditorEnzyklopaediePsychologieSoziale2016
und inproceedings

Selbstbestimmungstheorie nach Deci und Ryan (Autonomie, Fähigkeit, Zugehörigkeit)

===
To dive into the topic of human computer interaction 
===

Psychologie (was motiviert allgemein) -> übertragen auf die Mensch-Maschine-Interaktion = Human Computer Interaction (Wie agiert der Mensch mit dem Computer/der Maschine)

\subsection{Motivational Patterns}
\label{sec:theoryBc}
Motivational Patterns
-> hier auch Flow, ... aus bierhoffeditorEnzyklopaediePsychologieSoziale2016

\subsection{Gamification best practices and process TODO!?}
\label{sec:theoryBd}
Gamification Methods (Player Centered Design, Mechanics, ...)

\subsection{Gamification and motivational methods and patterns in business software TODO}
\label{sec:theoryBe}
(u.a. behavioral economics)

\subsection{Risks of Gamification}
\label{sec:theoryBf}
Risks of Gamification (u.a. Korrumpierungseffekt, "Overfitting???", Source: Does Gamification Work? — A Literature Review of Empirical Studies on
Gamification)


\newpage
