%%%--------------------------------%%%
%%% Theory
%%%--------------------------------%%%
\newpage
\section{Gamification}
\label{sec:theoryB}

The following chapters aim is to clarify the main theory behind human motivation, gamification and the corresponding patterns and methods. Therefore first of all the term Gamification is defined and explained (chapter \ref{sec:theoryBa}), furthermore there is an introduction to human motivation (chapter \ref{sec:theoryBb}) and motivational patterns (chapter \ref{sec:theoryBc}). TODO REST

\subsection{Definition Gamification}
\label{sec:theoryBa}

The term gamification is defined by Kumar and Herger as follows:

\begin{fquote}[Gamification {\protect\cite[p. 8]{inproceedings}}]
	Gamification is the application of game design principles and mechanics to
	non-game environments. It attempts to make technology more inviting by encouraging users to engage in desired behaviors and by showing the path to mastery.
	From a business viewpoint, gamification is using people’s innate enjoyment of play.
\end{fquote}

Further definitions are TODO

\cite[p. 8]{inproceedings}

More umfassender: Motivation -> next chapter(s)

Gamification aims to motivate the user to do something. That's why the next chapter provides a more comprehensive introduction on motivation.

\subsection{Motivation}
\label{sec:theoryBb}

The game design principles and mechanics which are used in the context of gamification are a specialization of motivational patterns used in Human Computer Interaction. \cite[p. 59]{inproceedings}

Therefore this chapter provides an introduction into the underlying psychology of motivation with the different types of motivation (extrinsic and intrinsic), behavioral psychology and behavioral economics.

\paragraph*{Psychology of motivation}
Human motivation is one of the main areas of psychology. Some questions which arouse are: What motivates humans for doing something? What intentions do they pursue with their doing? Which activities are a pleasure for them? \cite[p. 1]{bierhoffeditorEnzyklopaediePsychologieSoziale2016}

Mainly there are two types of motivation: extrinsic and intrinsic motivation. 
On the one hand intrinsic motivation is based on an internal drive to do something. The human is doing this task for their own. Possible motivational factors are gained autonomy, mastery or freedom. \cite[p. 2, 3, 4]{bierhoffeditorEnzyklopaediePsychologieSoziale2016}, \cite[p. 60, 61]{inproceedings}

Deci describes intrinsic motivation as follows: "One is said to be intrinsically motivated to perform an activity when he receives no apparent rewards except the activity itself." \cite[p. 105]{deciEffectsExternallyMediated1971}

On the other hand extrinsic motivation is based on motivational factors from the outside, such as money, throphys or the comparison with others through (for example with points, levels or leaderboards). \cite[p. 2, 3, 4]{bierhoffeditorEnzyklopaediePsychologieSoziale2016}, \cite[p. 60, 61]{inproceedings}

=======================\newline
TODO: Warum diese Unterscheidung im Folgenden relevant???

TODO: mit den unteren Theorien (B.J. Fogg, Selbstbestimmungstheorie) erklären warum der Mensch Dinge macht (aus Motivation)

TODO: weiter nach bierhoffeditorEnzyklopaediePsychologieSoziale2016
und inproceedings

Selbstbestimmungstheorie nach Deci und Ryan (Autonomie, Fähigkeit, Zugehörigkeit)

B.J. Fogg’s Behavior Model

\paragraph*{Behavioral psychology}


\paragraph*{Behavioral economics}

Behavioral economics explores, which effects affect economic decisions. In general whenever a resource (e.g. time, money) is reached or lost it is the consequence of a decision. So behavioral economics could also be seen as the theory behind decision making. Moreover in the context of Human Computer Interaction whenever a user interacts with an application 
lots of decisions are made. Engaging application design tries to include aspects of behavioral economics to influence the users decisions to spend more time in the application. 
Human decisions could be rational or irrational. Rational decisions are made to reach a concrete aim such as happiness and can be logically explained. Irrational decisions are not necessarily comprehensible.  Nevertheless irrational decisions can be triggered by external influences. For example people tend to use memberships, even if they doesn't profit (e.g. injured people go to the gym to use the membership).
Referring to the relationship between behavioral economics and application design the application can be designed to trigger the user to made an irrational decision (e.g. spend more time inside the application than needed). \cite[p. 19]{lewisIrresistibleAppsMotivational2014}


Patterns which motivate the user to do something by using the theoretical background of motivation, behavioral psychology and behavioral economics are described in the following chapter \ref{sec:theoryBc}

====================\newline
Psychologie (was motiviert allgemein) -> übertragen auf die Mensch-Maschine-Interaktion = Human Computer Interaction (Wie agiert der Mensch mit dem Computer/der Maschine)

\subsection{Motivational Patterns}
\label{sec:theoryBc}

The theoretical concepts above are used in various motivational patterns.

TODO: Liste der im folgenden vorgestellten Motivational Patterns und Erklärung warum diese ausgewählt wurden
Für eine weiterfühtrende Beschreibung weiterer sei auf ... verwiesen

\begin{itemize}
	\item Flow \newline
	TODO: Beschreibung \cite[p. 19, 20, 21]{bierhoffeditorEnzyklopaediePsychologieSoziale2016}
	TODO: Die weiteren Punkte von da bierhoffeditorEnzyklopaediePsychologieSoziale2016
	\item Feedback loops
\end{itemize}

Motivational Patterns
-> hier auch Flow, ... aus bierhoffeditorEnzyklopaediePsychologieSoziale2016

\subsection{Gamification best practices and process TODO!?}
\label{sec:theoryBd}
Gamification Methods (Player Centered Design, Mechanics, ...)

\subsection{Gamification and motivational methods and patterns in business software TODO}
\label{sec:theoryBe}
(u.a. behavioral economics)

\subsection{Risks of Gamification}
\label{sec:theoryBf}
Risks of Gamification (u.a. Korrumpierungseffekt, "Overfitting???", Source: Does Gamification Work? — A Literature Review of Empirical Studies on
Gamification)


\newpage
