%%%--------------------------------%%%
%%% Theory
%%%--------------------------------%%%
\section{Risk Management}
\label{sec:theoryA}
There are plenty of reasons for projects to fail and frequently even large companies and organizations experience costly failures of big projects \cite{dwivediResearchInformationSystems2015}. Projects are often defined as failed, when they cannot meet time or budget constraints or do not fulfill the pre-defined requirements. However, this definition is not useful in every context \cite{debakkerDoesRiskManagement2010}. IT projects often follow agile management techniques allowing for changes in the pre-defined scopes \cite{kusay-merkleAgilesProjektmanagementIm2018}. To allow for a wider understanding of what IT-project failure means Lyytien and Hirchheim group such failures into four different categories \cite{lyytinenInformationSystemsFailures1988}: 
\begin{itemize}
	\item Correspondence: Not meeting the pre-defined objectives
	\item Process: Exceeding time or budget restrains
	\item Interaction: Lack of end-user engagement
	\item Expectation: Inability to meet stakeholder’s expectations
	
\end{itemize}	
Analogous to the variety of ways in which a project can be defined as being unsuccessful there are many reasons which can lead to any such failure. Plenty research has been done to investigate the causes of project failure \cite{guptaSystematicLiteratureReview2018}.  Events that lead to project failure can be understood as risks. Islam \cite{islamSoftwareDevelopmentRisk2011} provides the following definition for risks in an IT context:
\begin{fquote}[TODO: Unterschrift {\protect\cite[S. 16]{TODO: Quelle}}]
	“Software risk, is defined as, the possibilities of suffering a loss such as budget or schedule over-runs, customer dissatisfaction, poor quality and passive customer involvement due to an undesirable event and its consequences during the life cycle of the project”
\end{fquote}
Many such risk factors have been identified in the literature by now. The following risks are an excerpt from lists collect via literature reviews by Whitney and Daniels [7] and Tesch et al [8]. 

\begin{table}[H]
	\centering
	\begin{tabular}{p{0.5\textwidth}p{0.5\textwidth}}
	Whitney and Daniels  & Tech et al. \\
	\begin{itemize}	\item	Lack of top management commitment to the project	
		\item	Failure to gain user commitment	
		\item	Misunderstanding the requirements	
		\item	Lack of adequate user involvement	
		\item	 Lack of required knowledge/skills in the project personnel	
		\item	Changing scope/objectives	
		\item	Introduction of new technology	
		\item	Failure to manage end user expectations	
		\item	Insufficient/inappropriate staffing	
		\item	Poor project management	
		\item	Excessive schedule pressure	
		\item	Lack of technical specifications	\end{itemize}
		&
		\begin{itemize}	\item	 Personnel shortfall and straining computer science abilities	
			\item	Unrealistic schedules and budgets	
			\item	System functionality	
			\item	Requirements management	
			\item	Resource usage and performance	
			\item	Personnel management	
			\item	Unrealistic project goals and objectives	
			\item	 Poor project team composition	
			\item	 Project management and control problems	
			\item	Problematic technology base/infrastructure		 
		\end{itemize}	
		\end{tabular}
\end{table}
To counter such risk factors risk management practices can be integrated into the overall project management. Risk management serves to identify risks, analyze them and to address them to minimize the damage these risks could do to project [8]. \\

Risk management is a process that should be initiated early in the project lifecycle to enable proactive handling of threats [6]. In general the cycle of risk management involves the following steps: Identification, Analysis, Response/Treatment and monitoring and control [6], [8], [9], [10]. The steps should be undertaken at the beginning of the project and updated whenever changes occur. There are different and more detailed variations [8], however for the purpose of this paper the general model will be assessed in more detail. \\
\begin{figure}[htbp] 
	\centering
	\includegraphics[width=1.0\textwidth]{Content/Theory/riskmanagmentcycle.png}
	\caption{Title}
	\cite{bibkey}
	\label{fig:riskmanagmentcycle}
\end{figure}
The different steps of the process serve different purposes and have different side effects. Risk identification helps to create awareness and to initiate action in general. It is also a phase during which the project team and stakeholders can share their concerns regarding the project and clarify their expectations to form a common view [9].\\

To actually perform the identification different techniques can be used. Two commonly used ones are the checklist and brainstorming [6], [9]. Checklists rely on past experience to identify known risk factory which are applicable to the project at hand. Another variant of procedure is to use a questionnaire instead which covers characteristics of the project to find specifically corresponding risks. Brainstorming is ideally done together with project stakeholders to gain different perspectives. Risk identification techniques are not mutually exclusive and combinations may result in more comprehensive results [6].\\

Risk analysis serves to create acceptance of the previously identified risks as well as to indicate their impact [9]. During this phase the likelihood of risk occurrence and the impact are estimated. This can be done in a qualitative manner by assigning ordinal values for both dimensions. The scales for likelihood can for example go from rare to almost certain. Impact can be described from low to catastrophic. Such estimates are subjective and my produce unclear results however trying to apply quantitative techniques can be unreliable as well since estimations based on past data may not be applicable anymore in a rapidly changing environment such as IT [6].\\

Risk response planning serves to reduce threats and to enhance opportunities [8]. Dealing with risks can be approached in different manners. Measurements can be defined to either avoid or prevent the risks or to deal with the impact should the risk occur. Another alternative can be to simply accept the risks or to outsource the risks [6]. Another practice used is to assign risk owners to establish clear responsibility for later control efforts [11].\\

Risk control serves to initiate action on the monitored risks and to direct action [6]. Monitoring the risks enables responding to changes via new cycles of the risk management process as well as triggering the measurements defined during the previous phase if necessary [8].  Techniques employed during this phase can be risk audits, trend analysis or regular status meetings [6]. 

