%%%--------------------------------%%%
%%% Introduction
%%%--------------------------------%%%
\chapter{Introduction}
\label{ch:introduction}

Software projects in the past have struggled to succeed. The Standish Group \cite{thestandishgroupinternationalincChaosReport20152015} has compiled a database of over 25,000 software projects over the years 2011 to 2015. They found that only about 30 percent of these projects were successful without any curtailments. The larger the projects were, the less often they turned out successful. This issue has been discussed in literature for decades now yet the problem of IT project failure still persists motivating this work. \cite{thestandishgroupinternationalincChaosReport20152015}

There are plenty of reasons for projects to fail and frequently even large companies and organizations experience costly failures of big projects \cite{dwivediResearchInformationSystems2015}. Events that lead to project failure can be understood as risk factors. To approach these risk factors, proper risk management practices should be established \cite{teschITProjectRisk2007}.

In practice, risk management is often disregarded \cite{kwakProjectRiskManagement2004}. To address this problem, the authors of this paper employ gamification techniques in the development of a project risk management tool, in order to influence the motivation of project managers and their teams to counter negligence in the course of a project’s lifetime.

The goals and approaches taken are:
\begin{itemize}
	\item Research on the needed theoretical background.
	\item Evaluation of the attitude of project managers regarding project risk management through a survey.
	\item Designing an application for project risk management by taking 
	the domain knowledge, the results of the survey, gamification and technical foundations into account. The designed application is visualized by a clickable prototype.
	\item Evaluation of the designed application by user interviews based on the clickable prototype.
	\item Feasibility demonstration of the implementation on the basis of a prototypical implementation resulting in a \ac{MVP}.
\end{itemize}

Therefore this paper is structured as follows:
The theoretical background is sketched in chapter \ref{ch:theorie}. Based on this, the conception can be found in chapter \ref{ch:domain} including a survey with potential users, use cases and requirements, the gamification conception and a clickable prototype. The implementation of the \acs{MVP} itself is described in chapter \ref{ch:implementation}. In the end, our work and its outcome is concluded and an outlook (\ref{ch:conclusionoutlookB}) is given.

%Latex Cheat Sheet:
%Bildquelle mit Seite:% \cite[S.1-42]{bibkey}
%Quelle normal: % \cite{freiknechtBigDataPraxis2018}

%Bilder normalerweise:
%\begin{figure}[htbp] 
%	\centering
%	\includegraphics[width=1.0\textwidth]{logos/dhbwLogo.png}
%	\caption{Title}
%	\cite{}
%	\label{fig:label01}
%\end{figure}
%Bild über den Seitenrand vergrößern und mittig ausrichten:
%\begin{figure}[htbp] 
%	\centering
%	\makebox[\textwidth]{ 
%		\includegraphics[width=1.0\textwidth]{logos/dhbwLogo.png}}
%	\caption{Title}
%	\cite{}
%	\label{fig:labelName}
%\end{figure}
%
%Fancy quotes:
%\begin{fquote}[text {\protect\cite[S. 16]{}}]
%	cite
%\end{fquote}
%
%Tabelle:
%\begin{table}[H]
%	\centering
%	\begin{tabular}{c|c|c}
%		\toprule 
%		Spalte1Titel  & Spalte2Titel & Spalte3Titel \\ \midrule 
%		1 & 3 &  5  \\ %\cmidrule{1-4}%
%		2 & 4 &  6  \\ %\cmidrule{1-4}%
%	\end{tabular}
%	\caption{Unterschrift}
%	\label{tab:label42}
%\end{table}

%\begin{center}
%	\begin{minipage}{\textwidth}
%		\lstinputlisting[
%			style = fancy,
%			language = python,
%			label = alg:Declaration,
%			caption = Title
%			]{sourcecode.txt}
%	\end{minipage}
%\end{center}

%Acronym erste Verwendung: \ac{API} \\
%Acronym n-te Verwendung: \ac{API} \\
%Acronym Kurz: \acs{API} \\
%Acronym Lang: \acl{API} \\
%Hallo\footnote{Ich bin eine Fußnote} du da

%\bigskip

%Referenz zur Einleitung (A): \autoref{ch:einleitung}, Seite \pageref{ch:einleitung} \\
%Referenz zur Einleitung (B): Kapitel \ref{ch:einleitung}, Seite \pageref{ch:einleitung} \\

%\cite[S. 265]{hurwitzLibGuidesLaTeXBibTeX}

%----------- Latex Vorlagen -----------