%%%--------------------------------%%%
%%% Introduction
%%%--------------------------------%%%
\chapter{Introduction}
\label{ch:introduction}

Software projects in the past have struggled to succeed. The Standish Group \cite{thestandishgroupinternationalincChaosReport20152015} has compiled a database of over 25,000 software projects over the years 2011 to 2015. They found that only about 30 percent of these projects were successful without any curtailments. The larger the projects were the less often they turned out successful. This issue has been discussed in the literature for decades now yet the problem of IT project failure still persists motivating this work.

There are plenty of reasons for projects to fail and frequently even large companies and organizations experience costly failures of big projects \cite{dwivediResearchInformationSystems2015}. Events that lead to project failure can be understood as risk factors. To counter such risk factors risk manage-ment practices can be integrated into the overall project management. Risk management serves to identify risks, analyze them and to address them to minimize the damage these risks could do to project \cite{teschITProjectRisk2007}.

Different studies have been undertaken to evaluate the effect of risk management on project success \footnote{Karel de Bakker, Albert Boonstra, and HansWortmann (2010), Otniel Didraga (2013), \cite{kwakProjectRiskManagement2004}, \cite{peixotoProjectRiskManagement2014}}. The results vary regarding which part of risk management or which tools and techniques contribute to projects success but some sort of positive impact is reported from all of them.

However, in practice risk management is often neglected \cite{kwakProjectRiskManagement2004}. Even if there is initial investment into risk management during the planning phase of a project there are tendencies to let efforts slide once the project is running and time pressure picks up \cite{peixotoProjectRiskManagement2014}.

To address this problem, the authors of this paper employ gamification techniques in the development of a project risk management tool and aim at influencing the motivation of project managers and their teams to counter negligence in the course of a project’s lifetime. During this paper, an application that in the future can be used to explore the effectiveness of gamification in the context of project risk management, will be developed (\ref{ch:implementation}). Before that, the theoretical background (\ref{ch:theorie}) and a concept (\ref{ch:domain}), including all use cases and requirements, will be analysed. In the end, the course of this paper is discussed (\ref{ch:discussionA}) and an outlook (\ref{ch:conclusionoutlookB}) is given.

%Latex Cheat Sheet:
%Bildquelle mit Seite:% \cite[S.1-42]{bibkey}
%Quelle normal: % \cite{freiknechtBigDataPraxis2018}

%Bilder normalerweise:
%\begin{figure}[htbp] 
%	\centering
%	\includegraphics[width=1.0\textwidth]{logos/dhbwLogo.png}
%	\caption{Title}
%	\cite{}
%	\label{fig:label01}
%\end{figure}
%Bild über den Seitenrand vergrößern und mittig ausrichten:
%\begin{figure}[htbp] 
%	\centering
%	\makebox[\textwidth]{ 
%		\includegraphics[width=1.0\textwidth]{logos/dhbwLogo.png}}
%	\caption{Title}
%	\cite{}
%	\label{fig:labelName}
%\end{figure}
%
%Fancy quotes:
%\begin{fquote}[text {\protect\cite[S. 16]{}}]
%	cite
%\end{fquote}
%
%Tabelle:
%\begin{table}[H]
%	\centering
%	\begin{tabular}{c|c|c}
%		\toprule 
%		Spalte1Titel  & Spalte2Titel & Spalte3Titel \\ \midrule 
%		1 & 3 &  5  \\ %\cmidrule{1-4}%
%		2 & 4 &  6  \\ %\cmidrule{1-4}%
%	\end{tabular}
%	\caption{Unterschrift}
%	\label{tab:label42}
%\end{table}

%\begin{center}
%	\begin{minipage}{\textwidth}
%		\lstinputlisting[
%			style = fancy,
%			language = python,
%			label = alg:Declaration,
%			caption = Title
%			]{sourcecode.txt}
%	\end{minipage}
%\end{center}

%Acronym erste Verwendung: \ac{API} \\
%Acronym n-te Verwendung: \ac{API} \\
%Acronym Kurz: \acs{API} \\
%Acronym Lang: \acl{API} \\
%Hallo\footnote{Ich bin eine Fußnote} du da

%\bigskip

%Referenz zur Einleitung (A): \autoref{ch:einleitung}, Seite \pageref{ch:einleitung} \\
%Referenz zur Einleitung (B): Kapitel \ref{ch:einleitung}, Seite \pageref{ch:einleitung} \\

%\cite[S. 265]{hurwitzLibGuidesLaTeXBibTeX}

%----------- Latex Vorlagen -----------