%%%--------------------------------%%%
%%% Introduction
%%%--------------------------------%%%
\chapter{Introduction}
\label{ch:einleitung}

context, motivation, aims, purpose, ..

Latex Cheat Sheet:
Bildquelle mit Seite:% \cite[S.1-42]{bibkey}
Quelle normal: % \cite{freiknechtBigDataPraxis2018}

Bilder normalerweise:
\begin{figure}[htbp] 
	\centering
	\includegraphics[width=1.0\textwidth]{logos/dhbwLogo.png}
	\caption{Title}
	\cite{}
	\label{fig:label01}
\end{figure}
Bild über den Seitenrand vergrößern und mittig ausrichten:
\begin{figure}[htbp] 
	\centering
	\makebox[\textwidth]{ 
		\includegraphics[width=1.0\textwidth]{logos/dhbwLogo.png}}
	\caption{Title}
	\cite{}
	\label{fig:labelName}
\end{figure}

Fancy quotes:
\begin{fquote}[text {\protect\cite[S. 16]{}}]
	cite
\end{fquote}

Tabelle:
\begin{table}[H]
	\centering
	\begin{tabular}{c|c|c}
		\toprule 
		Spalte1Titel  & Spalte2Titel & Spalte3Titel \\ \midrule 
		1 & 3 &  5  \\ %\cmidrule{1-4}%
		2 & 4 &  6  \\ %\cmidrule{1-4}%
	\end{tabular}
	\caption{Unterschrift}
	\label{tab:label42}
\end{table}

\begin{center}
	\begin{minipage}{\textwidth}
		\lstinputlisting[
			style = fancy,
			language = python,
			label = alg:Declaration,
			caption = Title
			]{sourcecode.txt}
	\end{minipage}
\end{center}

%Acronym erste Verwendung: \ac{API} \\
%Acronym n-te Verwendung: \ac{API} \\
%Acronym Kurz: \acs{API} \\
%Acronym Lang: \acl{API} \\
%Hallo\footnote{Ich bin eine Fußnote} du da

%\bigskip

%Referenz zur Einleitung (A): \autoref{ch:einleitung}, Seite \pageref{ch:einleitung} \\
%Referenz zur Einleitung (B): Kapitel \ref{ch:einleitung}, Seite \pageref{ch:einleitung} \\

%\cite[S. 265]{hurwitzLibGuidesLaTeXBibTeX}

%----------- Latex Vorlagen -----------