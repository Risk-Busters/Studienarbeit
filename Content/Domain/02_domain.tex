%%%--------------------------------%%%
%%% Domain
%%%--------------------------------%%%
\newpage
\section{Software Specification}
\label{sec:domainB}

\subsection{Purpose and Scope}
\label{sec:domainBa}
What is the software for and what is not part of the project (I'm integrating the vision part here because I don't think we need to state why we describe requirements)

\subsection{Functionalities}
\label{sec:domainBb}
Basically just the \ac{UC}

\subsubsection{Overall Use Case Diagram}
\label{sec:domainBba}


\newpage
% UC1 ====================================================
\subsubsection{Use Case Specification: \ac{UC}1 User CRUD}
\label{sec:domainBbb}

\paragraph*{Description}\mbox{}\\
This use case specifies how user accounts are being created, updated, deleted and are used for login and logout.
These fields are required for the user account creation:

\begin{itemize}
	\vspace{-3mm}
	\setlength\itemsep{-1em}
	\item e-mail adress (String)
	\item password (String)
\end{itemize}
Additionaly the user can set the following field while updating his account:
\begin{itemize}
	\vspace{-3mm}
	\setlength\itemsep{-1em}
	\item language (enum)
\end{itemize}

\paragraph*{Screenshots}\mbox{}\\
Insert screenshots and shortly explain what can be seen
\begin{figure}[h] 
	\centering
	\includegraphics[width=0.1\textwidth]{Content/Domain/placeholder.png}
	\caption{Use Case X: Detail}
	\label{fig:useCaseXDetailY}
\end{figure}

\newpage
\paragraph*{Basic Flow} \mbox{}\\
\noindent
New Account:
\begin{itemize}
	\vspace{-3mm}
	\setlength\itemsep{-1.5em}
	\item The visitor clicks the "register" button and fills in the fields mentioned in the brief description.
	\item Then clicks on the "register" button below the form to send the input to the server.
	\item If the visitors input fulfills the criteria of a username, email and password the account is being created on the server and a conformation e-mail is send to the given e-mail address.
\end{itemize}
The account itself can only be used after by clicking on the provided link in the conformation e-mail.

\spacing{1}

\noindent
Update Account: 
\begin{itemize}
	\vspace{-3mm}
	\setlength\itemsep{-1em}
	\item The user is logged in and clicks on the "settings" button.
	\item After updating the the fields the user wants to change, the input is sent to the server.
	\item If the updated fields fulfill their criteria, those fields are updated on the server.
\end{itemize} 

\noindent
Delete Account:
\begin{itemize}
	\vspace{-3mm}
	\setlength\itemsep{-1em}
	\item The user is logged in and clicks on the "settings" button.
	\item The user then clicks on the "delete my profile" button. 
	\item After typing in the users password, the request to delete the profile can be send to the server by clicking on the "delete" button.
	\item If the given password match the user's password the account is being deleted on the server and user will be logged out.
\end{itemize}

\noindent
Login:
\begin{itemize}
	\vspace{-3mm}
	\setlength\itemsep{-1em}
	\item The visitor already created an account and wants to log in.
	\item Therefor the "login" button must be clicked and the e-mail and password fields have to be filled in.
	\item The visitor can then click the "login" button below the form.
	\item If the e-mail and password match an entry of the list of registered users, the visitor will be logged in and can act as an user.
\end{itemize}

\noindent
Logout:
\begin{itemize}
	\vspace{-3mm}
	\setlength\itemsep{-1em}
	\item The user is logged in and wants to log out.
	\item By clicking the "logout" button the client removes all information used to authenticate to the server and forces the user interface to render without any user specific information fetched from the server.
\end{itemize}
The user is now able to login again.

\spacing{1.5}

\newpage
\subparagraph{Activity Diagram}\mbox{}\\
\begin{figure}[h]
	\centering
	\includegraphics[width=0.9\textwidth]{Content/Domain/UC1UserCRUDactivitydiagram.png}
	\caption{Activity Diagram \ac{UC}1 User CRUD}
	\label{fig:activityDiagramUC1}
\end{figure}

\paragraph*{Alternative Flows}\mbox{}\\
\spacing{1}

\noindent
Invalid input: 
\begin{itemize}
	\vspace{-3mm}
	\setlength\itemsep{-1em}
	\item The user / visitor fills in the required fields (username, e-mail, password, language).
	\item If one field does not meet its criteria (e.g. an e-mail should contain the @-sign), which is validated by the server after submitting the input, the user / visitor will be informed next to the input fields.
\end{itemize} 
In this case the user account won't be created nor updated.

\noindent
Invalid user credentials:
\begin{itemize}
	\vspace{-3mm}
	\setlength\itemsep{-1em}
	\item If a visitor wants to login into an existing user account or if an account should be deleted by a user, the user accounts credentials have to be submitted (e-mail and password for login, password only for deletion).
	\item If the account credentials do not match any entry on the server, the user / visitor will be informed next to the input fields.
\end{itemize}
In this flow the visitor won't be logged in and neither will the account be deleted.

\spacing{1.5}
\paragraph*{Special Requirements and Preconditions}\mbox{}\\
Different flows require different preconditions.
\begin{enumerate}
	\vspace{-3mm}
	\setlength\itemsep{-1em}
	\item If a visitor wants to log into an user account has to exist or created beforehand.
	\item If a user wants to log out, the user has to be logged in.
	\item If a user wants to update or delete the account,the user has to be logged in, too.
\end{enumerate}

\paragraph*{Postconditions and Persistance}\mbox{}\\
Mentionable postconditions are:
\begin{enumerate}
	\vspace{-3mm}
	\setlength\itemsep{-1em}
	\item After deleting an account, the account can not be restored and therefor not being used anymore.
	\item Furthermore logging in allows a user to then access user related content from the server and logging out hides all user related content respectively.
\end{enumerate}

\noindent
The persistence guidelines are: 
\newline
\noindent
Creating, updating and deleting a user account or logging in by filling in the required fields and then submitting the input leads to a POST request to the server. If the fields fulfill their criteria, the information will be persisted on the server. Logging out will only delete the users authentication information on the client side and will not send a request to the server.


\newpage
% UC2 ====================================================
\subsubsection{Use Case Specification: \ac{UC}2 Project Access Management CRUD}
\label{sec:domainBbc}

\paragraph*{Description}\mbox{}\\
This use case specifies how projects are being created, updated, deleted and are used for grouping a pool of users and risks.
The fields in the list below are required for the project creation:

\begin{itemize}
	\vspace{-3mm}
	\setlength\itemsep{-1em}
	\item projetc name (String)
	\item project description (String)
	\item start date (Date)
	\item end date (Date)
\end{itemize}
Besides the required fields the project owner can set the following field while updating a project:
\begin{itemize}
	\vspace{-3mm}
	\setlength\itemsep{-1em}
	\item project members (Strings: Usernames, E-Mails)
\end{itemize}

\paragraph*{Screenshots}\mbox{}\\
Insert screenshots and shortly explain what can be seen
\begin{figure}[h] 
	\centering
	\includegraphics[width=0.1\textwidth]{Content/Domain/placeholder.png}
	\caption{Use Case X: Detail}
	\label{fig:useCaseXDetailY}
\end{figure}

\newpage
\paragraph*{Basic Flow} \mbox{}\\
\noindent
New Project:
\begin{itemize}
	\vspace{-3mm}
	\setlength\itemsep{-1.5em}
	\item The user is currently on the project page and clicks the "new project" button and fills in the required fields.
	\item Then clicks on the "create" button to send the input to the server.
	\item If the users input fulfills the criteria of a name, description and time span, the project will be created on the server and be available in the project list.
\end{itemize}

\spacing{1}

\noindent
Update Project: 
\begin{itemize}
	\vspace{-3mm}
	\setlength\itemsep{-1em}
	\item The user opened the detailed project page by clicking on it on the project list and then clicks the "edit" button.
	\item After updating the fields the user, to be more precise the project owner, intended to change, those updated fields are send to the server by clicking the "save" button.
	\item If the updated fields fulfill their criterias, those fields are updated on the server and in the user interface.
\end{itemize} 
Adding users while updating a project is the basis to work on risks together as described in chapter \ref{sec:domainBbd}

\noindent
Delete Project:
\begin{itemize}
	\vspace{-3mm}
	\setlength\itemsep{-1em}
	\item The project owner opened the detailed project page by clicking on it on the project list and then clicks the "edit" button.
	\item The owner then clicks on the "delete" button. 
	\item After typing in the users password, the request to delete the project will be send to the server once the "delete" button is clicked.
	\item If the given password match the owner's password the project, including all related risks, is being deleted on the server and removed from the project list in the user interface.
\end{itemize}

\spacing{1.5}

\newpage
\subparagraph{Activity Diagram}\mbox{}\\
\begin{figure}[h]
	\centering
	\includegraphics[width=0.9\textwidth]{Content/Domain/UC2ProjectCRUDactivitydiagram.png}
	\caption{Activity Diagram \ac{UC}2 Project CRUD}
	\label{fig:activityDiagramUC2}
\end{figure}

\paragraph*{Alternative Flows}\mbox{}\\
\spacing{1}

\noindent
Invalid input: 
\begin{itemize}
	\vspace{-3mm}
	\setlength\itemsep{-1em}
	\item Like chapter \ref{sec:domainBbb} "Invalid Input", but for project creations and updates.
\end{itemize}

\noindent
Invalid user credentials:
\begin{itemize}
	\vspace{-3mm}
	\setlength\itemsep{-1em}
	\item Like chapter \ref{sec:domainBbb} "Invalid user credentials", but for project deletion.
\end{itemize}

\spacing{1.5}
\paragraph*{Special Requirements and Preconditions}\mbox{}\\
Generally, the user who created the project is automatically the project owner.
\begin{enumerate}
	\vspace{-3mm}
	\setlength\itemsep{-1em}
	\item The user has to be logged in to create a project. 
	\item The user has to be the owner of a project to edit it.
\end{enumerate}

\paragraph*{Postconditions and Persistance}\mbox{}\\
Mentionable postconditions are:
\begin{enumerate}
	\vspace{-3mm}
	\setlength\itemsep{-1em}
	\item When deleting a project, the project including all related risks can not be restored.
	\item After creating a project and adding users, users will be able to access it and create risks within.
\end{enumerate}

\noindent
The persistence guidelines are: 
\newline
\noindent
All mentioned basic flows above create POST requests to the server and additionally, if the fields meet their criterias the data will be persisted.

\newpage
% UC3 ====================================================
\subsubsection{Use Case Specification: \ac{UC}3 Risk CRUD}
\label{sec:domainBbd}

\paragraph*{Description}\mbox{}\\
This use case allows users to create, read, update and delete risks. 
A risk consists of the following fields:
\begin{itemize}
	\vspace{-3mm}
	\setlength\itemsep{-1em}
	\item name (String)
	\item description (String)
\end{itemize}
Following fields are filled later and are not part of the input form:
\begin{itemize}
	\vspace{-3mm}
	\setlength\itemsep{-1em}
	\item probability of occurence (Enum)
	\item impact (Enum)
	\item risk factor (Enum)
	\item response (Objects)  
	\item person in charge (User)
	\item public risk (boolean)
\end{itemize}

\paragraph*{Screenshots}\mbox{}\\
tbd: Insert screenshots and shortly explain what can be seen
\begin{figure}[h] 
	\centering
	\includegraphics[width=0.1\textwidth]{Content/Domain/placeholder.png}
	\caption{\ac{UC}3 Risk CRUD: TODO}
	\label{fig:label3}
\end{figure}

\newpage
\paragraph*{Basic Flow} \mbox{}\\
\noindent
Creating a risk:
\begin{itemize}
	\vspace{-3mm}
	\setlength\itemsep{-1.5em}
	\item  When the user clicks the "+" button at the project overview page.
	\item Then the screen for adding a new risk is opened.
	\item When the risk form is filled by the user.
	\item And the user clicks on the "Propose risk" button.
	\item Then the risk is synced with the server.
\end{itemize}

\spacing{1}
\noindent
Reading a risk:
\begin{itemize}
	\vspace{-3mm}
	\setlength\itemsep{-1em}
	\item The user is on the project overview site with all project risks.
	\item By clicking on a risk a detail risk view is opened.
	\item For exiting the risk detail view a return button ("Close" button) is clicked.
\end{itemize}

\noindent
Updating a risk: 
\begin{itemize}
	\vspace{-3mm}
	\setlength\itemsep{-1em}
	\item The user is on the project overview site with all project risks.
	\item By clicking on a risk a detail risk view is opened.
	\item On the detail view there is a pen button, enabling editing and changing the "Close" button to a "Save" button.
	\item When clicking the "Save" button the changes are syncronized with the server.
\end{itemize} 

\noindent
Deleting a risk:
\begin{itemize}
	\vspace{-3mm}
	\setlength\itemsep{-1em}
	\item The user is on the project overview site with all project risks.
	\item By clicking on a risk a detail risk view is opened.
	\item By clicking a "Delete" button the risk is deleted. This behavior is changed in UC6 Risk Discussion described in chapter \ref{sec:domainBbg}.
\end{itemize}

\spacing{1.5}
 
\subparagraph{Activity Diagram}\mbox{}\\
\begin{figure}
	\centering
	\includegraphics[width=0.9\textwidth]{Content/Domain/UC3RiskCRUDactivitydiagram.png}
	\caption{Activity Diagram \ac{UC}3 Risk CRUD}
	\label{fig:activityDiagramUC3}
\end{figure}

\paragraph*{Alternative Flows}\mbox{}\\
n/a

\paragraph*{Special Requirements and Preconditions}\mbox{}\\
The preconditions for this use case are:
\begin{enumerate}
	\vspace{-3mm}
	\setlength\itemsep{-1em}
	\item A project exists.
	\item The user is member of the project.
	\item The user has clicked the "+" button at the project overview page to add a new risk.
\end{enumerate}

\paragraph*{Postconditions and Persistance}\mbox{}\\
The postconditions for this use case are:
\begin{enumerate}
	\vspace{-3mm}
	\setlength\itemsep{-1em}
	\item The risk is immediately part of the projects risk table (this behavior is changed within UC6 Risk Discussion described in chapter \ref{sec:domainBbg}).
\end{enumerate}

\noindent
The persistence guidelines are: 
\newline
\noindent
The risk form was completely or partly filled by the user. When the user tries to leave the page now, there should be a prompt for exiting. When the risk form is filled out and the button "Propose risk" is clicked a POST request syncs the status with the server.

\newpage
% UC4 ====================================================
\subsubsection{Use Case Specification: \ac{UC}4 Project risk overview}
\label{sec:domainBbe}

\paragraph*{Description}\mbox{}\\
Describe the functionality

\paragraph*{Screenshots}\mbox{}\\
Insert screenshots and shortly explain what can be seen
\begin{figure}[h] 
	\centering
	\includegraphics[width=0.1\textwidth]{Content/Domain/placeholder.png}
	\caption{Use Case X: Detail}
	\label{fig:label4}
\end{figure}

\paragraph*{Basic Flow} \mbox{}\\

Describe the most common path through this use case

\subparagraph{Activity Diagram}\mbox{}\\
\begin{figure}[h]
	\centering
	\includegraphics[width=0.1\textwidth]{Content/Domain/placeholder.png}
	\caption{Activity Diagram Use Case X}
	\label{fig:label44}
\end{figure}

\paragraph*{Alternative Flows}\mbox{}\\
What can go wrong :D

\paragraph*{Special Requirements and Preconditions}\mbox{}\\
Where does the user come from, what does he have to do before he gets here

\paragraph*{Postconditions and Persistance}\mbox{}\\
What has changed and how do we make sure the change persists?

\newpage
% UC5 ====================================================
\subsubsection{Use Case Specification: \ac{UC}5 Risk Pool}
\label{sec:domainBbf}

\paragraph*{Description}\mbox{}\\
Describe the functionality

\paragraph*{Screenshots}\mbox{}\\
Insert screenshots and shortly explain what can be seen
\begin{figure}[h] 
	\centering
	\includegraphics[width=0.1\textwidth]{Content/Domain/placeholder.png}
	\caption{Use Case X: Detail}
	\label{fig:label5}
\end{figure}

\paragraph*{Basic Flow} \mbox{}\\

Describe the most common path through this use case

\subparagraph{Activity Diagram}\mbox{}\\
\begin{figure}[h]
	\centering
	\includegraphics[width=0.1\textwidth]{Content/Domain/placeholder.png}
	\caption{Activity Diagram Use Case X}
	\label{fig:label55}
\end{figure}

\paragraph*{Alternative Flows}\mbox{}\\
What can go wrong :D

\paragraph*{Special Requirements and Preconditions}\mbox{}\\
Where does the user come from, what does he have to do before he gets here

\paragraph*{Postconditions and Persistance}\mbox{}\\
What has changed and how do we make sure the change persists?

\newpage
% UC6 ====================================================
\subsubsection{Use Case Specification: \ac{UC}6 Risk Discussion}
\label{sec:domainBbg}

\paragraph*{Description}\mbox{}\\
The process of adding risks to a project is based on three steps:
\begin{enumerate}
	\vspace{-3mm}
	\setlength\itemsep{-1em}
	
	\item A project member proposes a risk (name and description). (UC3 Risk CRUD defined in chapter \ref{sec:domainBbd}).
	\item A specified number of project members review the proposed risk. After the review process the risk is ready for discussion.
	\item The whole project team discusses and defines:
	\begin{itemize}
		\vspace{-3mm}
		\setlength\itemsep{-1em}
		
		\item probability of occurence
		\item impact
		\item risk factor (defined by probability of occurence and impact)
		\item response(s)
		\item person in charge
	\end{itemize}
\end{enumerate}

\paragraph*{Screenshots}\mbox{}\\
tbd: Insert screenshots and shortly explain what can be seen
\begin{figure}[h] 
	\centering
	\includegraphics[width=0.1\textwidth]{Content/Domain/placeholder.png}
	\caption{\ac{UC}6 Risk Discussion: TODO}
	\label{fig:label6}
\end{figure}

\newpage

\paragraph*{Basic Flow} \mbox{}\\

\begin{enumerate}
	\vspace{-3mm}
	\setlength\itemsep{-1em}
	
	\item A project member proposes a risk (name and description as defined in UC3 Risk CRUD chapter \ref{sec:domainBbd})
	\begin{itemize}
		\vspace{-3mm}
		\setlength\itemsep{-1em}
		
		\item The proposed risk is visible in the section proposed risks at the project overview page ("Proposed risks").
		\item All members of the project receive a notification (in their activity stream) about the new proposed risk and the request to review it.
	\end{itemize}
	
	\item A specified number of project members review the proposed risk. After the review process the risk is ready for discussion.
	\begin{itemize}
		\vspace{-3mm}
		\setlength\itemsep{-1em}
		
		\item When the specified number of project members have positively reviewed the risk, it is visible in the section ("Risks open to discussion").
		\item The project owner is able to manually start an estimation session for the risk (not recommended). The recommended way is to wait until a specified number of risks open for estimation are collected. Then an estimation session for all risks is automatically started.
	\end{itemize}
		
	\item The risk is estimated by the whole project team:
	\begin{itemize}
		\vspace{-3mm}
		\setlength\itemsep{-1em}
		
		\item All project members receive a notification (in their activity stream) to estimate the risk in terms of (probability of occurrence, impact) and to add response(s). Probability of occurrence and impact are mandatory fields, response(s) are optional.
		\item The risk factor is determined by the probability of occurrence and the impact.
		\item It is checked if at least one response is defined. If not the project members are notified to add a response.
		\item In the last step the person in charge is defined. Therefore a notification where the user is able to sign up for being responsible for the risk is sent to all project members.  
	\end{itemize}

	\item Finally the risk appears at the project risk table ("Project risks").
	
\end{enumerate}

\subparagraph{Activity Diagram}\mbox{}\\
\begin{figure}[!hp]
	\centering
	\includegraphics[width=0.8\textwidth]{Content/Domain/UC6RiskDiscussion.png}
	\caption{Activity Diagram \ac{UC}6 Risk Discussion}
	\label{fig:label66}
\end{figure}

\newpage

\paragraph*{Alternative Flows}\mbox{}\\

\noindent
Proposed risk receives negative reviews:
\begin{itemize}
	\vspace{-3mm}
	\setlength\itemsep{-1em}
	
	\item A project member negatively reviews a risk.
	\item The reviewing person adds feedback for the risk which is sent to the risk owner.
	\item The risk owner can edit or delete the proposed risk.
	\item When the risk is deleted it is removed from the section proposed risks.
	\item When the risk is edited the review process data is removed and the message for reviewing is sent again.
\end{itemize}

\noindent
Project members doesn't react to their notifications to:
\begin{enumerate}
	\vspace{-3mm}
	\setlength\itemsep{-1em}
	
	\item review a risk: When the needed amount of reviews is not achieved the notification is sent again.
	\item estimate a risk: When the needed amount of estimations is not achieved the notification is sent again.
	\item define a person in charge: When no person volunteers the notification is sent again. For finding a risk person in charge fast the Gamification concept Challenge is used.
\end{enumerate}

\paragraph*{Special Requirements and Preconditions}\mbox{}\\
The preconditions for this use case are:
\begin{enumerate}
	\vspace{-3mm}
	\setlength\itemsep{-1em}
	
	\item  A project exists.
	\item The user is member of the project.
	\item The user has proposed a risk.
\end{enumerate}

\paragraph*{Postconditions and Persistance}\mbox{}\\
The postconditions for this use case are:
\begin{enumerate}
	\vspace{-3mm}
	\setlength\itemsep{-1em}
	
	\item The proposed risk was reviewed and estimated. 
	\item The risk contains all relevant data.
	\item The risk is visible as part of the projects risk table.
\end{enumerate}

\newpage
% UC7 ====================================================
\subsubsection{Use Case Specification: \ac{UC}7 Risk Monitoring}
\label{sec:domainBbh}

\paragraph*{Description}\mbox{}\\
Describe the functionality

\paragraph*{Screenshots}\mbox{}\\
Insert screenshots and shortly explain what can be seen
\begin{figure}[h] 
	\centering
	\includegraphics[width=0.1\textwidth]{Content/Domain/placeholder.png}
	\caption{Use Case X: Detail}
	\label{fig:label7}
\end{figure}

\paragraph*{Basic Flow} \mbox{}\\

Describe the most common path through this use case

\subparagraph{Activity Diagram}\mbox{}\\
\begin{figure}[h]
	\centering
	\includegraphics[width=0.1\textwidth]{Content/Domain/placeholder.png}
	\caption{Activity Diagram Use Case X}
	\label{fig:label77}
\end{figure}

\paragraph*{Alternative Flows}\mbox{}\\
What can go wrong :D

\paragraph*{Special Requirements and Preconditions}\mbox{}\\
Where does the user come from, what does he have to do before he gets here

\paragraph*{Postconditions and Persistance}\mbox{}\\
What has changed and how do we make sure the change persists?

\newpage
% UC8 ====================================================
\subsubsection{Use Case Specification: \ac{UC}8 Risk Response Management}
\label{sec:domainBbi}

\paragraph*{Description}\mbox{}\\
Describe the functionality

\paragraph*{Screenshots}\mbox{}\\
Insert screenshots and shortly explain what can be seen
\begin{figure}[h] 
	\centering
	\includegraphics[width=0.1\textwidth]{Content/Domain/placeholder.png}
	\caption{Use Case X: Detail}
	\label{fig:label8}
\end{figure}

\paragraph*{Basic Flow} \mbox{}\\

Describe the most common path through this use case

\subparagraph{Activity Diagram}\mbox{}\\
\begin{figure}[h]
	\centering
	\includegraphics[width=0.1\textwidth]{Content/Domain/placeholder.png}
	\caption{Activity Diagram Use Case X}
	\label{fig:label88}
\end{figure}

\paragraph*{Alternative Flows}\mbox{}\\
What can go wrong :D

\paragraph*{Special Requirements and Preconditions}\mbox{}\\
Where does the user come from, what does he have to do before he gets here

\paragraph*{Postconditions and Persistance}\mbox{}\\
What has changed and how do we make sure the change persists?

\newpage
% UC9 ====================================================
\subsubsection{Use Case Specification: \ac{UC}9 Project Initialization}
\label{sec:domainBbj}

\paragraph*{Description}\mbox{}\\
Describe the functionality

\paragraph*{Screenshots}\mbox{}\\
Insert screenshots and shortly explain what can be seen
\begin{figure}[h] 
	\centering
	\includegraphics[width=0.1\textwidth]{Content/Domain/placeholder.png}
	\caption{Use Case X: Detail}
	\label{fig:label9}
\end{figure}

\paragraph*{Basic Flow} \mbox{}\\

Describe the most common path through this use case

\subparagraph{Activity Diagram}\mbox{}\\
\begin{figure}[h]
	\centering
	\includegraphics[width=0.1\textwidth]{Content/Domain/placeholder.png}
	\caption{Activity Diagram Use Case X}
	\label{fig:label10}
\end{figure}

\paragraph*{Alternative Flows}\mbox{}\\
What can go wrong :D

\paragraph*{Special Requirements and Preconditions}\mbox{}\\
Where does the user come from, what does he have to do before he gets here

\paragraph*{Postconditions and Persistance}\mbox{}\\
What has changed and how do we make sure the change persists?

\newpage
% UC10 ====================================================
\subsubsection{Use Case Specification: \ac{UC}10 Activity Stream}
\label{sec:domainBbk}

\paragraph*{Description}\mbox{}\\
A user should be able to see the latest activity (see chapter \ref{sec:theoryBc}) in the given enviroment.

\paragraph*{Screenshots}\mbox{}\\
Insert screenshots and shortly explain what can be seen
\begin{figure}[h] 
	\centering
	\includegraphics[width=0.1\textwidth]{Content/Domain/placeholder.png}
	\caption{Use Case X: Detail}
	\label{fig:label1}
\end{figure}

\paragraph*{Basic Flow} \mbox{}\\

\spacing{1}

\noindent
Activity Stream: 
\begin{itemize}
	\vspace{-3mm}
	\setlength\itemsep{-1em}
	\item The user opens the home page where the activity stream can be found.
	\item Within the activity stream the user can then find the latest activities (examples mentioned above).
	\item Depending on the activity the user has the chance to interact with it and will be redirected to the specific item (e.g. a user can click on a new risks shown in his activity stream to open it).
\end{itemize} 

\spacing{1.5}

\newpage
\subparagraph{Activity Diagram}\mbox{}\\
\begin{figure}[h]
	\centering
	\includegraphics[width=0.9\textwidth]{Content/Domain/UC10ActivityStream.png}
	\caption{Activity Diagram \ac{UC}10 Activity Stream}
	\label{fig:label11}
\end{figure}

\paragraph*{Alternative Flows}\mbox{}\\
N/A

\paragraph*{Special Requirements and Preconditions}\mbox{}\\
As the activity contains personalized information, the following preconditions have to be fulfilled.
\begin{enumerate}
	\vspace{-3mm}
	\setlength\itemsep{-1em}
	\item The user has to be logged in.
	\item Alternatively a visitor can login to be redirect to the personal home page.
\end{enumerate}

\paragraph*{Postconditions and Persistance}\mbox{}\\
N/A

\newpage
% UC11 ====================================================
\subsubsection{Use Case Specification: \ac{UC}11 Progress Indicator}
\label{sec:domainBbl}

\paragraph*{Description}\mbox{}\\
Describe the functionality

\paragraph*{Screenshots}\mbox{}\\
Insert screenshots and shortly explain what can be seen
\begin{figure}[h] 
	\centering
	\includegraphics[width=0.1\textwidth]{Content/Domain/placeholder.png}
	\caption{Use Case X: Detail}
	\label{fig:label8}
\end{figure}

\paragraph*{Basic Flow} \mbox{}\\

Describe the most common path through this use case

\subparagraph{Activity Diagram}\mbox{}\\
\begin{figure}[h]
	\centering
	\includegraphics[width=0.1\textwidth]{Content/Domain/placeholder.png}
	\caption{Activity Diagram Use Case X}
	\label{fig:label12}
\end{figure}

\paragraph*{Alternative Flows}\mbox{}\\
What can go wrong :D

\paragraph*{Special Requirements and Preconditions}\mbox{}\\
Where does the user come from, what does he have to do before he gets here

\paragraph*{Postconditions and Persistance}\mbox{}\\
What has changed and how do we make sure the change persists?

TODO: INSERT USE CASES CONTEXT

\subsection{Requirements}
\label{sec:domainBc}
Couldn't find a better translation for "Anspruch" :/
Describes which 'Ansprüche' we have regarding the following categories:
\subsubsection{Usability}
\label{sec:domainBca}
\subsubsection{Reliability}
\label{sec:domainBcb}
\subsubsection{Performance}
\label{sec:domainBcc}
\subsubsection{Supportability}
\label{sec:domainBcd}

\subsection{Design Constraints}
\label{sec:domainBd}
Where does our software run? Where does it not? Which functions cannot be covered and why? Which conditions are required to use the software? Stuff like that

\subsection{Interfaces}
\label{sec:domainBe}
\subsubsection{User Interfaces}
\label{sec:domainBea}
Describe the views the software will have
\subsubsection{Further Interfaces}
\label{sec:domainBeb}
Software, Hardware and Communcation Interfaces... expand subsubsections if necessary
