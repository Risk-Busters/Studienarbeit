%%%--------------------------------%%%
%%% Domain
%%%--------------------------------%%%

\section{User Feedback}
\label{sec:DomainA}
To validate the conclusions we drew from literature research we also gathered potential user feedback to guide our development efforts. First we conducted a short survey to figure out which phases of the risk management process and which platforms we should pay special attention to. We then developed an interactive mock-up to illustrate the core functionality of our application and did some user interviews based on that. We used the feedback to determine the focus areas and features of the application.
\subsection{Survey}
\label{sec:DomainAa}
The survey was conducted among project managers in two big German IT companies. The first sample was collected in December 2019 and amounted to ten answers. The second sample was gathered in January 2020 and contains of 18 project managers. Due to the small sample size our results are not generalizable. To draw any scientific conclusion the findings would have to be verified with a larger sample. However the results are still useful for guiding our development efforts.

We asked the PMs about their previous project experiences in terms of risk managment practice and risks encountered. We also included questions about the importance of different risk managment phases and the usefullness of tool support for them. Finally we gathered information about the platforms they used and on which they would want a risk managment tool. The full questionnaire can be found in the appendix.

The majority of the PMs we surveyed did undertake risk managment efforts for their projects which they oversaw during the last year. Those who did not mostly cited project size as their reason, however some said they usually didn't do it and two even thought it not useful at all as can be seen in \ref{fig:label20} and \ref{fig:label21}.
\begin{figure}[H]
	\centering
	\includegraphics[width=0.5\textwidth]{Assets/survey_results/Q1.png}
	\caption{Survey question 1 results}
	\label{fig:label20}
\end{figure}
\begin{figure}[H]
	\centering
	\includegraphics[width=0.7\textwidth]{Assets/survey_results/Q2.png}
	\caption{Survey question 2 results}
	\label{fig:label21}
\end{figure}

\subsection{Interviews}
\label{sec:DomainAb}
For the interviews we managed to secure further support from the company surveyed in December. Three project managers agreed to test our mock-up prototype and give us feedback. Two of the PMs were male, one was female. All three were provided with a convertible device and requested to explore the prototype while thinking out loud. We asked them to speak out whatever came to their minds while using the mock-ups and to pose any questions that came up. Afterwards we asked them for their overall assessment of the tool, which parts of it they deemed useful and where it was lacking.

All project managers rated the tool as useful for risk management efforts though for different reasons. Two of them said it was helpful in focusing their activities. One said that usually risk management was more of a minor activity that happened on the side. The other described the initally discussed pattern of risk management happening at the beginning of the project and then being neglected, forgotton or tedious to come back to. He expected that activity tracking and easy notification of team members via delibarte push notifications could actually turn risk managment into a process instead of a one-time activity. This supports our focus on motivation and long term commitment.

The third PM was more skeptical of these kinds of features. He said that with some or perhaps more considerable effort he could build such a process into his task managing tool and thus preferred not having to manage two tools. However, he saw much use in the risk pool feature which collects and persists the experience of many projects and thus would provide valuable guidance. The risk pool was also deemed useful by the other two PMs. We thus conclude that the risk pool is pracitcable way to adress the need of documentation and knowledge persistance identified in \ref{sec:theoryAd}.

All PMs also appreciated that risk evaluation was a group task and not determined by individual opinions. The fact that adjustment and re-opening of risk evaluation was possible after the inital risk assessment.

All three PMS requested more visual reporting aid to gain a clearer view of project health status. We take from it, that the activity graph does not sufficently cover the need for visualization and that further reporting graphics should be devised.

Furthermore some functionalities or positioning was not understood, which we will work on improving in the actual application. Also many small additional features were suggested which are listed in the appedix for future expansion.
