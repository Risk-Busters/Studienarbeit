%%%--------------------------------%%%
%%% Conception
%%%--------------------------------%%%

\section{Gamification}
\label{sec:domainC}

\spacing{1.5}

The Gamification conception is based on the Gamification Design Process described in chapter \ref{sec:theoryBd}. Therefore the following chapters deal with the Player (chapter \ref{sec:domainCa}), the Mission (chapter \ref{sec:domainCb}), Mechanics (chapter \ref{sec:domainCc}) and the evaluation (chapter \ref{sec:domainCd}).

\subsection{Player Personas}
\label{sec:domainCa}

By the initial survey the application's target group was figured out (chapter \ref{sec:DomainA}). Based on this Player Personas for the following three groups were created.

\begin{itemize}
	\item project manager
	\item product manager
	\item team member
\end{itemize}

TODO: Personas aufbauend auf Bild in Theorie

\subsection{Mission}
\label{sec:domainCb}

By defining the mission we aim to have a specific vision in mind, from which the following steps can be derived. Like described in the theory chapter \ref{sec:theoryBd} the mission is defined. 
First of all the current situation is considered (chapter \ref{sec:theoryA}). As target business outcome we would like to lower the number of projects failing because of not considered risk management. This should be achieved by the development of a gamified application for project risk management.
Our mission is to increase the user engagement and support habit building in terms of project risk management. Especially the by project managers demanded part of risk response management and monitoring (chapter \ref{sec:DomainA}) should support the project managers as best as possible.

\subsection{Mechanics}
\label{sec:domainCc}

This chapter describes the mechanics conception which Motivational Design Patterns from chapter \ref{sec:theoryBc} are used.
Following mechanics are used and described in detail:
\begin{itemize}
	\item Activity Stream and Notifications
	\item Growth, Specialization—Badge and Increased Responsibility
	\item Self defined challenges
	\item Praise and Rewards (Score)
	\item Collaborative form of leaderboards
	\item Journey
	\item Social Feedback/Feedback loops
	\item Task Queue
	\item Further used patterns
\end{itemize}

\paragraph*{Activity Stream and Notifications}
The activity stream represents the start page and is a key component of the application. Through it the user is notified about relevant changes. Furthermore the sent messages can motivate the user to 

Push notifications

\paragraph*{Growth, Specialization—Badge and Increased Responsibility}

\paragraph*{Self defined challenges}

\paragraph*{Praise and Rewards (Score)}

\paragraph*{Collaborative form of leaderboards}

\paragraph*{Journey}

\paragraph*{Social Feedback/Feedback loops}

\paragraph*{Task Queue}

\paragraph*{Further used patterns}

\begin{itemize}
	\item Broadcast (e.g. Risks are shared between users)
	\item Predictable Results
	\item State Preservation
	\item Organization of Information
	\item Personalization (e.g. Risk Pool with Recommendation engine)
	\item Reporting (e.g. risks can be reported)
	\item Search (e.g. Risk Pool is searchable)
\end{itemize}


\subsection{Evaluation}
\label{sec:domainCd}
Evaluation
%To measure if methods of gamification and motivalinal patterns influence the user's behavior -> Tracking and A/B-Test
%Version A: gamified
%Version B: not gamified