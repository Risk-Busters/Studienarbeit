%%%--------------------------------%%%
%%% Conclusion & Outlook
%%%--------------------------------%%%
\chapter{Zusammenfassung und Ausblick}
\label{ch:resumee}

\section{Zusammenfassung}
\label{sec:fazit}

Ziel dieser Arbeit war es, Erfahrungen zu sammeln, inwiefern inspirierende Produktempfehlungen durch einen Trend-Produkt-Recommender von den Kunden durch eine erhöhte Wahrnehmung honoriert werden. Dieses Ziel wurde auf Basis der in Kapitel \ref{sec:ziele} definierten Feinziele erreicht:
\begin{itemize}
	\item Entwicklung einer ersten Version eines Trend-Produkt-Recommenders \checkmark
	\item Inbetriebnahme des Recommenders im Rahmen eines A/B-Tests innerhalb der Produktionsumgebung des dm-Onlineshops \checkmark
	\item Erhebung von Daten des A/B-Tests im Produktivbetrieb \checkmark
	\item Auswertung der erhobenen Daten des A/B-Tests \checkmark
\end{itemize}

Im Rahmen dieser Projektarbeit wurde ein direkter Kundenmehrwert geschaffen, welchen die Kunden direkt auf ihrer \glqq Meine Produkte\grqq-Seite im dm-Onlinshop erleben können. Die dem Kunden präsentierten Produktempfehlungen stellen hierbei eine völlig neue Art der inspirierenden Produktempfehlungen dar. 

Den Service, dass neue Trends in der Filiale über besondere Aufsteller besonders präsentiert werden und somit ins Blickfeld der Kunden geraten wird diesen nun auch im Onlineshop auf ihrer \glqq Meine Produkte\grqq-Seite  geboten. Der große Vorteil, den die Version im Onlineshop bietet, ist die kundenspezifische Personalisierung. Diese kundenspezifische Personalisierung ist in der Filiale nicht möglich. 

Diese Projektarbeit stellt somit einen weiteren Baustein zu Verfügung, um dem Kunden im Onlineshop ein personalisiertes Shopping-Erlebnis zur Verfügung zu stellen und diesen somit mit dm zu verbinden.

\section{Ausblick}
\label{sec:ausblick}

Aufbauend auf dieser Arbeit kann der Trend-Produkt-Recommender mit folgenden weiterführenden Arbeiten weiterentwickelt und dann jeweils mit der bestehenden Version des Recommenders im Rahmen eines A/B-Tests verglichen werden:

\begin{itemize}
		\item Zur besseren Auswertung aller Kacheln im Vergleich können Trackingdaten für die noch fehlenden Kacheln erhoben werden.
		\item Die Daten der Suche des dm-Onlineshops können für den Recommender genutzt werden.
		\item Suchdaten von Google Trends und weiteren Suchmaschinen können  genutzt werden, um Trends zu erkennen und zu den Trends passende Produkte zu empfehlen.
		\item Es kann der Ansatz der saisonalen Dekomposition genutzt werden, um die Daten über den Zeitverlauf in eine saisonale Komponente und eine Trend Komponente zu zerlegen.
		\item Um den Einfluss der Sequenzeffekte genauer zu betrachten kann die Reihenfolge der Kachel im Rahmen eines A/B-Tests variiert werden.
\end{itemize}

