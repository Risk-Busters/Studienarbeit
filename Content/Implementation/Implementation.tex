%%%--------------------------------%%%
%%% Implementation
%%%--------------------------------%%%
\chapter{Implementation}
\label{ch:implementation}
\spacing{1.5}

This chapter will give an insight into what has been implemented and is functional yet. Due to the overall available time, the authors of this thesis focused on building a \ac{MVP}. In general, an \acs{MVP} is useful approach to learn about the users needs by only providing what is needed to run the application with its basic features \cite[p. 3]{keshlafModelPrototypeTool2000}

First of all, an overview about infrastructure related use cases is given. These include the UC1 User CRUD (\ref{sec:domainBbb}) and the UC2 Project Access Management (\ref{sec:domainBbc}).

Secondly, all risk related use cases are covered. The following use cases are part of this section: UC3 Risk CRUD (\ref{sec:domainBbd}), UC4 Adjustment (\ref{sec:domainBbe}), UC5 Pool (\ref{sec:domainBbf}), UC6 Discussion (\ref{sec:domainBbg}), UC7 Monitoring (\ref{sec:domainBbh}) and UC8 Response Management (\ref{sec:domainBbi}).

Right after, the implemented gamification mechanics (\ref{sec:domainCc}) and the dedicated use cases UC9 Project Initialization (\ref{sec:domainBbj}), UC10 Activity Stream (\ref{sec:sec:domainBbk}) and UC11 Progress Indicator (\ref{sec:domainBbl}) are discussed.

In the end, the design approach of this \acs{MVP} is briefly elucidated.

Furthermore, this chapter does not describe the software's architecture. One can refer to chapter \ref{sec:DomainC} for more information about the architecture and requirements.
%%%--------------------------------%%%
%%% Implementation N
%%%--------------------------------%%%
\foreach \i in {01,02,03,04,05,06,07,08,09,10,...,99} {%
	\edef\FileName{Content/Implementation/\i_implementation}%
	\IfFileExists{\FileName}{%
		\input{\FileName}
	}
	{%
		% File does not exist
		
	}
}%foreach