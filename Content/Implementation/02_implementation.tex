%%%--------------------------------%%%
%%% Implementation
%%%--------------------------------%%%

\section{Risk Use Cases}
\label{sec:implementationRisks}

Due to the coherence between most of the risk use cases, they will be demonstrated by one larger example. First of all, the Risk CRUD (\ref{sec:domainBbd}) which includes the creation and editing of a risk.  When opening the detailed view of a project, the user will be able to propose a risk by providing a title and a more detailed description. After proposing, every project member has the chance to like this proposed risk and after reaching a specific amount of likes, the risk will then be discussable. All of this is additionally part of the Risk Discussion use case (\ref{sec:domainBbg}), but is missing the ability to provide negative textual feedback. The implementation of both use cases had a very high priority as the existence of risks, including their status change, is essential for additional defined use cases where the existence of risks is mandatory like the Risk Monitoring (\ref{sec:domainBbh}) or the Risk Pool (\ref{sec:domainBbf}). Figure TODO XXX shows a proposed risk with two likes.

TODO: Proposed Risk with 2 likes.

The risk will now be displayed beneath the "Discuss" tab within the detailed project view from where users are able to provide an estimation as seen in figure TODO XXX.

TODO: Screenshot Disucssion window

After providing enough estimations, a median will be calculated and the project risk is ready to be finalized. In this step, one user has to assign the risks to themselves for what rewards will be distributed (\ref{sec:implementationRisks}) and multiple responses have to be defined. The latter is part of the Risk Response Management use case (\ref{sec:domainBbi}).

TODO: Screenshot finalization

If the mentioned steps have been fulfilled, the risk will be final and be displayed beneath the "Final" tab which opens automatically when opening the detailed project view.

 Missing from the MVP, though highly demanded (\ref{sec:DomainAa}), is the ability to monitor risks with the help of notifications and a due date as described in the corresponding use case (\ref{sec:domainBbh}). The reason for its absence from the MVP, as well for the other upcoming missing risk use cases, is the lack of time during development of this thesis and, as described in the beginning of this section, that they depend on other use cases that had to be implemented in first place. Prioritizing risks is also not possible yet but could be implemented like use case Project Risk Adjustment depicts (\ref{sec:domainBbe}). Also not part of the MVP is the Risk Pool (\ref{sec:domainBbf}) as currently a risk can’t leave its project and be published in a pool. Due to the high demand, the ability to monitor risks should be implemented firsts, followed by the Risk Pool which was another requested feature in the interviews (\ref{sec:DomainAb})
 
 To conclude, the Risk- and Risk Response Management are fully implemented, as well as the Risk Discussions in the broadest sense.