%%%--------------------------------%%%
%%% Implementation
%%%--------------------------------%%%

\section{Gamification Elements}
\label{sec:implementationGami}

Besides achievements as shown in section \ref{sec:implementationInfra}, the MVP has various further progress indicators which are part of the defined use case (\ref{sec:domainBbl}). First of all, users do receive points for desired behavior which can be seen in form of an activity graph for a single specific user or all project members together, secondly the project detail view shows a progress bar based on the projects temporal progress. Both can be seen in figure XXX. Described in the use case  but not implemented, due the lack of time and less relatedness as against "progress", is the on-boarding and scaffolding.

TODO: Screenshot project detail

Another approach to support the progress of a project is the implementation of a so-called task queue. To be specific, a user gets displayed further actions which make sense to continue working on after completing another task. One place where the application makes use of this is after proposing a project risk where options to propose another, review or discuss project risks are being displayed. The task queue, development wise, are simply links to different views of the application instead of more complex business logic which is why it is now part of the MVP as it did not occupied much development capacity.

Also, as part of the gamification elements and part of the defined use cases (\ref{sec:domainBbk}), an activity stream has been implemented. The application can keep track of events that occur during run time like the proposal of a new project risk. The activity stream can be found on the start page when logged in. An option to enable notifications is placed next to the activity stream which the user can use to allow its client to send notifications. Those notifications can be used in different use cases to inform the user about new events and make the application re-engageable (\ref{sec:theorieCa}).

TODO: Screenshot home screen

Not implemented is the use case Project Initialization (\ref{sec:domainBbj}) which could be implemented in the future to encourage project members to create the first risks within a project. The reason this use case is not implemented yet is that it first needs a functioning discussion process to makes sense and secondly a working notification system to implement the complete basic flow which lead to a lack of development capacity in the end.

Summarized, many mechanics as described the Gamification conception (\ref{sec:domainCc}) are part of the application: Activity Stream and Notifications, Badges, Praise and Rewards, Task Queues.