%%%--------------------------------%%%
%%% Implementation
%%%--------------------------------%%%

\section{Gamification Elements}
\label{sec:implementationGami}

Besides achievements as shown in TODO XXX, the MVP has various further progress indicators. First of all users do receive points for desired behavior which can be seen in form of an activity graph for a single specific user or all project members together, secondly the project detail view shows a progress bar based on the projects temporal progress. Both can be seen in figure XXX.

TODO: Screenshot project detail

Another approach to support the progress of a project is the implementation of a so-called task queue. To be specific, a user gets displayed further actions which make sense to continue working on after completing another task. One place where the application makes use of this is after proposing a project risk where options to propose another, review or discuss project risks are being displayed.

Described in the use case Progress Indicator (TODO XXX) but not implemented is the on-boarding and scaffolding. However, as part of the gamification elements, an activity stream has been implemented. The application can keep track of events that occur during run time like the proposal of a new project risk. The activity stream can be found on the start page when logged in. An option to enable notifications is placed next to the activity stream which the user can use to allow its client to send notifications. Those notifications can be used in different use cases to inform the user about new events and make the application re-engageable (TODO XXX).

TODO: Screenshot home screen

Not implemented is the use case Project Initialization (chapter XXX) which could be implemented in the future encourage project members to create the first risks within a project.