%%%--------------------------------%%%
%%% Glossary
%%%--------------------------------%%%
\chapter*{\nameofglossary}

%%%--------------------------------%%%
%%% Entries
%%%--------------------------------%%%

\begin{description}
	\item[Bartle's player type] The Bartle test of Psychology, developed by Richard Bartle, is used for the classification of users regarding their gaming behavior. There are four different user types: Achiever, Explorer, Socializer and Killer. Achievers want to achieve points and status through playing, Explorers are interested in discovering new aspects of the game, Socializers like the interaction with other players and Killers are an extreme form of Achievers, which get their energy from the defeat of others. Players can have character traits from all four player types. The tests aim is to find the major type for a player. \cite[p. 44, 45]{kumarGamificationWorkDesigning2013}
\end{description}

\begin{description}
	\item[Game design principles] Game design principles are central properties on which games are built. \cite[p. 8]{kumarGamificationWorkDesigning2013}
\end{description}

\begin{description}
	\item[Game mechanics] Game mechanics are typical methods which are used in games. \cite[p. 8]{kumarGamificationWorkDesigning2013}
	\item[JSON] \acl{JSON} is a widely used data interchange format and represents data in an universal friendly way. \cite[pp. 2-4]{bassettIntroductionJavaScriptObject2015}
	\item[REST / REST API] Describes how the Internet functions. A \acs{REST} \acs{API} on the other hand is a server that provides access to resources of the servers data model and functions. \cite[p. IX]{masseRESTAPIDesign2011}
	\item[SQL] The \acl{SQL} is used to interact with a relational database, such as putting data into it or querying data from the database. \cite[p. XIII]{molinaroSQLCookbookQuery2005}

\end{description}

